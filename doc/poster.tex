\documentclass[final]{beamer}
\mode<presentation>{\usetheme{I6pd2}}
\usepackage[latin1]{inputenc}
\usepackage{amsmath, amsthm, amssymb, latexsym}
\usepackage[orientation=landscape,size=a0,scale=1.4,debug]{beamerposter}
 
\title{\huge Integrative causal analysis of genetic and epigenetic data from a
large cohort of 392 individuals}
\author{Rosemary McCloskey and Sara Mostafavi}
\institute[UBC]{University of British Columbia, Vancouver, Canada}
\date[March 20th, 2015]{March 20th, 2015}

\newlength{\columnheight}
\setlength{\columnheight}{105cm}

\begin{document}
\begin{frame}
\begin{columns}

% Column 1
\begin{column}{0.33\textwidth}
\begin{block}{Motivation}
    \begin{itemize}
        \item genetic, epigenetic, and transcriptomic data provide snapshots of
            cellular processes
        \item usually one data type is studied at a time, in relation to a
            phenotype or disease
        \item \textbf{how do these data fit together?}
    \end{itemize}
    \begin{center}
        \includegraphics[scale=3]{figures/motivation}
    \end{center}
\end{block}

\begin{block}{Quantitative Trait Loci (QTLs)}
\begin{itemize}
    \item a QTL is a genetic locus correlated with a phenotype
\end{itemize}
\end{block}
\end{column}

% Column 2
\begin{column}{0.33\textwidth}
\begin{block}{Multiple Testing for QTLs}
\end{block}

\begin{block}{Removing Principal Components}
\end{block}

\begin{block}{Identifying multi-QTLs}
\end{block}
\end{column}

% Column 3
\begin{column}{0.33\textwidth}
\begin{block}{Bayesian Networks}
\end{block}

\begin{block}{Future Work}
\end{block}

\begin{block}{Acknowledgements}
\end{block}
\end{column}

\end{columns}
\end{frame}
\end{document}
